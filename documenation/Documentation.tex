\documentclass[reqno,10pt,a4paper]{article}
\usepackage{geometry}
\geometry{a4paper, portrait, left=3cm, right=3cm, top=3cm, bottom=3cm}

\usepackage{url}
\usepackage{amsmath}
\usepackage{amssymb}
\usepackage{amsthm}
\usepackage{latexsym}
\usepackage{wasysym}
\usepackage{tikz}

\usepackage{siunitx}
\sisetup{group-separator = \text{\,}}
\sisetup{group-minimum-digits = 3}

% Opening
\title{Magic Square of Perfect Squares}
\author{Adrian Suter}


\begin{document}
	
	\maketitle
	
	\section{The maths behind it}
	
	Each component $s_i$, $i = 1, 2, 3, \ldots, 9$ of the magic square has to be a perfect square number. Say $s_i = n_i^2$ for $n_i \in \mathbb{Z}$. From \cite{zimmermann2015} we know that the components need to be such that
	
	\begin{equation} \label{eq:conditionMod24}
		n_i^2 \equiv 1~(\textrm{mod}~24)
	\end{equation}
	
	Looking at the smallest perfect square numbers and their modulo 24 (see table \ref{perfectSquares}) we can see some sort of a pattern every 12-th integer. Indeed, as
	\begin{equation}
	(x + 12)^2 \mod 24 = (x^2 + 24x + 144) \mod 24
	\end{equation}
	for any integer $x$ and as
	\begin{equation}
	(24x + 144) \mod 24 = 0
	\end{equation}
	we can see that
	\begin{equation}
	x^2 \mod 24 = (x + 12)^2 \mod 24
	\end{equation}
	
	\begin{table}
			\begin{tabular}{|c|c|c||c|c|c|}
			\hline
			$x$ & $x^2$ & $x^2 \mod 24$ & $x$ & $x^2$ & $x^2 \mod 24$ \\
			\hline
			0 & 0 & 0 & 12 & 144 & 0 \\
			1 & 1 & 1 & 13 & 169 & 1 \\
			2 & 4 & 4 & 14 & 196 & 4 \\
			3 & 9 & 9 & 15 & 225 & 9 \\
			4 & 16 & 16 & 16 & 256 & 16 \\
			5 & 25 & 1 & 17 & 289 & 1 \\
			6 & 36 & 12 & 18 & 324 & 12 \\
			7 & 49 & 1 & 19 & 361 & 1 \\
			8 & 64 & 16 & 20 & 400 & 16 \\
			9 & 81 & 9 & 21 & 441 & 9 \\
			10 & 100 & 4 & 22 & 484 & 4 \\
			11 & 121 & 1 & 23 & 529 & 1 \\
			\hline
		\end{tabular}
		\caption{Some perfect square numbers and their modulo 24.} \label{perfectSquares}
	\end{table}
	
	For $x \in [0, 11]$ there are exactly four perfect squares fulfilling the condition \eqref{eq:conditionMod24}. Therefore we can build four infinite series $\mathbb{S}_j$, $j = 1, 2, 3, 4$ for which all members fulfill the condition. Let there be $t$ any integer, then
	\begin{eqnarray}
		\mathbb{S}_1 &=& 1 + 12 t = \left\{ 1, 13, 25, \ldots \right\} \\
		\mathbb{S}_2 &=& 5 + 12 t = \left\{ 5, 17, 29, \ldots \right\} \\
		\mathbb{S}_3 &=& 7 + 12 t = \left\{ 7, 19, 31, \ldots \right\} \\
		\mathbb{S}_4 &=& 11 + 12 t = \left\{ 11, 23, 35, \ldots \right\}
	\end{eqnarray}
	
	The square numbers of the members of all four sets are the only perfect square numbers that fulfill the condition \eqref{eq:conditionMod24}. There are no others.
	
	
	\section{The generator functions}
	
	The above four series can be generated using two generator functions $f_+(g)$ and $f_-(g)$ for any integer $g$.
	
	\begin{eqnarray}
		f_+ &=& 6 \cdot g + 1 \label{eq:generatorFunctionPlus} \\
		f_- &=& 6 \cdot g - 1 \label{eq:generatorFunctionMinus}
	\end{eqnarray}
	
	We will call $g$ the generator number.
	
	
	\section{Program usage}
	
	Once the program had been launched, it waits to receive on the standard input a generator string. The generator string consists of a generator number $g$ followed optionally by a \verb§+§ (default) or \verb§-§ character. By hitting enter, the program would start calculating the magic squares for the given generator string and at the end it would print a dash \verb§-§ to the standard output. As soon as this happened, the program is ready for the next generator string. If any valid or almost\footnote{More than six perfect square numbers} valid magic square had been detected, the program would write a file containing the most important information to the disk and prints the filename to the standard output.
	
	
	\section{Program flow}
	
	The last character of the generator string defines the generator function to be applied. Either it ends by \verb§-§, then the function \eqref{eq:generatorFunctionMinus} would be used or it ends by \verb§+§ (default), then the function \eqref{eq:generatorFunctionPlus} would be applied.
	
	Let $r$ be the calculated number. From now on we assume that the center magic square number is $s_5 = r^2$, therefore $n_5 = r$.
	
	To be continued...
	
	
	%%
	\bibliographystyle{plain}
	\bibliography{Library}

\end{document}