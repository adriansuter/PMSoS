\documentclass[reqno,10pt,a4paper]{article}
\usepackage{geometry}
\geometry{a4paper, portrait, left=3cm, right=3cm, top=3cm, bottom=3cm}

\usepackage{url}
\usepackage{amsmath}
\usepackage{amssymb}
\usepackage{amsthm}
\usepackage{latexsym}
\usepackage{wasysym}
\usepackage{tikz}
\usepackage[framemethod=TikZ]{mdframed}

\usepackage{siunitx}
\sisetup{group-separator = \text{\,}}
\sisetup{group-minimum-digits = 3}

\newenvironment{example}[2][]{%
\ifstrempty{#1}%
% if condition (without title)
{\mdfsetup{%
		frametitle={%
			\tikz[baseline=(current bounding box.east),outer sep=0pt]
			\node[anchor=east,rectangle,fill=blue!10]
			{\strut Example};}%
	}%
	% else condition (with title)
}{\mdfsetup{%
		frametitle={%
			\tikz[baseline=(current bounding box.east),outer sep=0pt]
			\node[anchor=east,rectangle,fill=blue!10]
			{\strut Example:~#1};}%
	}%
}%
% Both conditions
\mdfsetup{%
	innertopmargin=4pt,linecolor=blue!10,%
	linewidth=2pt,topline=true,%
	frametitleaboveskip=\dimexpr-\ht\strutbox\relax%
}	
\begin{mdframed}[]\relax}{%
\end{mdframed}}

% Opening
\title{The quest for $3 \times 3$ magic squares of perfect square numbers}
\author{Adrian Suter}


\begin{document}
	
	\maketitle
	
	\section{Introduction}
		
	A $3 \times 3$ magic square is defined as a square matrix of size $3$
	
	\begin{equation}
		\mathbf{S} = 
		\begin{pmatrix}
		s_1 & s_2 & s_3 \\
		s_4 & s_5 & s_6 \\
		s_7 & s_8 & s_9
		\end{pmatrix}
	\end{equation}
	where the components $s_i$, $i = 1, 2, 3, \ldots, 9$ are distinct positive integers and the sum of the integers in each row, column and diagonal is equal. We call that sum \emph{magic sum} $S$. Hence
	\begin{eqnarray}
		S &=& s_1 + s_2 + s_3 = s_4 + s_5 + s_6 = s_7 + s_8 + s_9 \\
		S &=& s_1 + s_4 + s_7 = s_2 + s_5 + s_8 = s_3 + s_6 + s_9 \\
		S &=& s_1 + s_5 + s_9 = s_3 + s_5 + s_7
	\end{eqnarray}
	
		
	In order to have a $3 \times 3$ magic square of perfect square numbers, each component $s_i$, $i = 1, 2, 3, \ldots, 9$ of the magic square $\mathbf{S}$ has to be a perfect square number. Say $s_i = n_i^2$ for $n_i \in \mathbb{Z}$. So
	
	\begingroup
	\renewcommand*{\arraystretch}{1.5}
	\begin{equation}
		\mathbf{S} = 
		\begin{pmatrix}
			n_1^2 & n_2^2 & n_3^2 \\
			n_4^2 & n_5^2 & n_6^2 \\
			n_7^2 & n_8^2 & n_9^2
		\end{pmatrix}
	\end{equation}
	\endgroup	
	
	
	The french mathematician \'{E}douard Lucas showed (CITE) that all $3 \times 3$ magic squares could be written in the following form.
	
	\begin{equation}
	\mathbf{S} = 
	\begin{pmatrix}
	c - b & c + (a + b) & c - a \\
	c - (a - b) & c & c + (a - b) \\
	c + a & c - (a + b) & c + b
	\end{pmatrix}
	\end{equation}
	for positive integers $a$, $b$ and $c$ with $0 < a < b < c - a$ and $b \neq 2a$.
	
	Let
	\begin{equation}
		d = a + b \text{ and } e = a - b \text{.}
	\end{equation}
	
	Therefore
	\begin{eqnarray}
		s_1 = n_1^2 &=& c - b \\
		s_2 = n_2^2 &=& c + d \\
		s_3 = n_3^2 &=& c - a \\
		s_4 = n_4^2 &=& c - e \\
		s_5 = n_5^2 &=& c \\
		s_6 = n_6^2 &=& c + e \\
		s_7 = n_7^2 &=& c + a \\
		s_8 = n_8^2 &=& c - d \\
		s_9 = n_9^2 &=& c + b
	\end{eqnarray}
	
	So we would actually search four arithmetic progressions around $c$
	\begin{eqnarray}
		\mathbb{P}_c(a) &=& \left\{ c - a, c, c + a \right\} = \left\{ n_3^2, n_5^2, n_7^2 \right\} \\
		\mathbb{P}_c(b) &=& \left\{ c - b, c, c + b \right\} = \left\{ n_1^2, n_5^2, n_9^2 \right\} \\
		\mathbb{P}_c(d) &=& \left\{ c - d, c, c + d \right\} = \left\{ n_8^2, n_5^2, n_2^2 \right\} \\
		\mathbb{P}_c(e) &=& \left\{ c - e, c, c + e \right\} = \left\{ n_4^2, n_5^2, n_6^2 \right\}
	\end{eqnarray}
	for which $d = a + b$ and $e = a - b$.

	\vspace{1em}
	
	From \cite{zimmermann2015} we know that the components need to be such that
	
	\begin{equation} \label{eq:conditionMod24}
		n_i^2 \equiv 1~(\textrm{mod}~24)
	\end{equation}
	
	Looking at the smallest perfect square numbers and their modulo 24 (see table \ref{perfectSquares}) we can see some sort of a pattern every 12-th integer. Indeed, as
	\begin{equation}
	(x + 12)^2 \mod 24 = (x^2 + 24x + 144) \mod 24
	\end{equation}
	for any integer $x$ and as
	\begin{equation}
	(24x + 144) \mod 24 = 0
	\end{equation}
	we can see that
	\begin{equation}
	x^2 \mod 24 = (x + 12)^2 \mod 24
	\end{equation}
	
	\begin{table}
			\begin{tabular}{|c|c|c||c|c|c|}
			\hline
			$x$ & $x^2$ & $x^2 \mod 24$ & $x$ & $x^2$ & $x^2 \mod 24$ \\
			\hline
			0 & 0 & 0 & 12 & 144 & 0 \\
			1 & 1 & 1 & 13 & 169 & 1 \\
			2 & 4 & 4 & 14 & 196 & 4 \\
			3 & 9 & 9 & 15 & 225 & 9 \\
			4 & 16 & 16 & 16 & 256 & 16 \\
			5 & 25 & 1 & 17 & 289 & 1 \\
			6 & 36 & 12 & 18 & 324 & 12 \\
			7 & 49 & 1 & 19 & 361 & 1 \\
			8 & 64 & 16 & 20 & 400 & 16 \\
			9 & 81 & 9 & 21 & 441 & 9 \\
			10 & 100 & 4 & 22 & 484 & 4 \\
			11 & 121 & 1 & 23 & 529 & 1 \\
			\hline
		\end{tabular}
		\caption{Some perfect square numbers and their modulo 24.} \label{perfectSquares}
	\end{table}
	
	For $x \in [0, 11]$ there are exactly four perfect squares fulfilling the condition \eqref{eq:conditionMod24}. Therefore we can build four infinite series $\mathbb{S}_j$, $j = 1, 2, 3, 4$ for which all members fulfill the condition. Let there be $t$ any integer, then
	\begin{eqnarray}
		\mathbb{S}_1 &=& 1 + 12 t = \left\{ 1, 13, 25, \ldots \right\} \\
		\mathbb{S}_2 &=& 5 + 12 t = \left\{ 5, 17, 29, \ldots \right\} \\
		\mathbb{S}_3 &=& 7 + 12 t = \left\{ 7, 19, 31, \ldots \right\} \\
		\mathbb{S}_4 &=& 11 + 12 t = \left\{ 11, 23, 35, \ldots \right\}
	\end{eqnarray}
	
	The square numbers of the members of all four sets are the only perfect square numbers that fulfill the condition \eqref{eq:conditionMod24}. There are no others.
	
	
	\section{The generator functions}
	
	The above four series can be generated using two generator functions $f_+(g)$ and $f_-(g)$ for any integer $g$.
	
	\begin{eqnarray}
		f_+(g) &=& 6 \cdot g + 1 \label{eq:generatorFunctionPlus} \\
		f_-(g) &=& 6 \cdot g - 1 \label{eq:generatorFunctionMinus}
	\end{eqnarray}
	
	We will call $g$ the generator number.
	
	
	\section{Program usage}
	
	Once the program had been launched, it waits to receive on the standard input a generator string. The generator string consists of a generator number $g$ followed optionally by a \verb§+§ (default) or \verb§-§ character. By hitting enter, the program would start calculating the magic squares for the given generator string and at the end it would print a dash \verb§-§ to the standard output. As soon as this happened, the program is ready for the next generator string. If any valid or almost\footnote{More than six perfect square numbers} valid magic square had been detected, the program would write a file containing the most important information to the disk and prints the filename to the standard output.
	
	
	\section{Program flow}
	
	The last character of the generator string defines the generator function to be applied. Either it ends by \verb§-§, then the function \eqref{eq:generatorFunctionMinus} would be used or it ends by \verb§+§ (default), then the function \eqref{eq:generatorFunctionPlus} would be applied.
	
	Let $n_5$ be this calculated number. Therefore the center cell of our magic square would become
	\begin{equation}
		s_5 = n_5^2
	\end{equation}

	In the program we would store $n_5$ in the variable \verb§number§ and $s_5$ in \verb§numberSquared§.
	
	\begin{example}{}
		Let the generator string be \verb§"141-"§. That means our program would calculate
		\begin{equation*}
			n_5 = f_-(141) = 6 \cdot 141 - 1 = \num{845}
		\end{equation*}
		and
		\begin{equation*}
			s_5 = \num{845}^2 = \num{714025} \text{.}
		\end{equation*}
	\end{example}
	
	Next we calculate all factor pairs for $n_5$, that is we build the set of factor pairs
	\begin{equation}
		\mathbb{P} = \left\{ (p_1, q_1), (p_2, q_2), \ldots, (p_K, q_K) \right\}
	\end{equation}
	where $K = |\mathbb{P}|$ and where
	\begin{equation}
		p_k \cdot q_k = n_5 ~\forall~ k \in [1, K] \subset \mathbb{Z} \text{.}
	\end{equation}
	
	\begin{example}{1}
		The set of factor pairs for our example would be
		\begin{equation}
			\mathbb{P} = \left\{ (13, 65), (5, 169), (1, 845) \right\}
		\end{equation}
	\end{example}

	To be continued...
	
	
	\begin{example}{Debug Program Output}
		\begin{verbatim}
		> 141-
		Input: 141, -1
		Number: 845, Number^2: 714025
		-- Factor Pairs --
		13 x 65
		5 x 169
		1 x 845
		-- Arithmetic Progressions --
		519841, 714025, 908209 | 194184
		508369, 714025, 919681 | 205656
		373321, 714025, 1054729 | 340704
		207025, 714025, 1221025 | 507000
		89401, 714025, 1338649 | 624624
		28561, 714025, 1399489 | 685464
		25, 714025, 1428025 | 714000
		-- Valid Combinations --
		(194184, 205656)...
		508369 1113865 519841
		725497 714025 702553
		908209 314185 919681
		(194184, 340704)...
		373321 1248913 519841
		860545 714025 567505
		908209 179137 1054729
		(194184, 507000)...
		207025 1415209 519841
		1026841 714025 401209
		908209 12841 1221025
		(194184, 624624) Skip as a + b >= c
		(194184, 685464) Skip as a + b >= c
		(194184, 714000) Skip as a + b >= c
		(205656, 340704)...
		373321 1260385 508369
		849073 714025 578977
		919681 167665 1054729
		(205656, 507000)...
		207025 1426681 508369
		1015369 714025 412681
		919681 1369 1221025
		ps,6,141M,1.result
		(205656, 624624) Skip as a + b >= c
		(205656, 685464) Skip as a + b >= c
		(205656, 714000) Skip as a + b >= c
		(340704, 507000) Skip as a + b >= c
		(340704, 624624) Skip as a + b >= c
		(340704, 685464) Skip as a + b >= c
		(340704, 714000) Skip as a + b >= c
		(507000, 624624) Skip as a + b >= c
		(507000, 685464) Skip as a + b >= c
		(507000, 714000) Skip as a + b >= c
		(624624, 685464) Skip as a + b >= c
		(624624, 714000) Skip as a + b >= c
		(685464, 714000) Skip as a + b >= c
		
		Content of the file ps,6,141M,1.result
		--------------------------------------
		845
		714025
		1 0 1 | 0 1 0 | 1 1 1
		207025 1426681 508369 | 1015369 714025 412681 | 919681 1369 1221025
		\end{verbatim}
	\end{example}

	
	
	
	%%
	\bibliographystyle{plain}
	\bibliography{Library}

\end{document}